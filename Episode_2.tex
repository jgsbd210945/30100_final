% Options for packages loaded elsewhere
\PassOptionsToPackage{unicode}{hyperref}
\PassOptionsToPackage{hyphens}{url}
\PassOptionsToPackage{dvipsnames,svgnames,x11names}{xcolor}
%
\documentclass[
  letterpaper,
  DIV=11,
  numbers=noendperiod]{scrartcl}

\usepackage{amsmath,amssymb}
\usepackage{iftex}
\ifPDFTeX
  \usepackage[T1]{fontenc}
  \usepackage[utf8]{inputenc}
  \usepackage{textcomp} % provide euro and other symbols
\else % if luatex or xetex
  \usepackage{unicode-math}
  \defaultfontfeatures{Scale=MatchLowercase}
  \defaultfontfeatures[\rmfamily]{Ligatures=TeX,Scale=1}
\fi
\usepackage{lmodern}
\ifPDFTeX\else  
    % xetex/luatex font selection
\fi
% Use upquote if available, for straight quotes in verbatim environments
\IfFileExists{upquote.sty}{\usepackage{upquote}}{}
\IfFileExists{microtype.sty}{% use microtype if available
  \usepackage[]{microtype}
  \UseMicrotypeSet[protrusion]{basicmath} % disable protrusion for tt fonts
}{}
\makeatletter
\@ifundefined{KOMAClassName}{% if non-KOMA class
  \IfFileExists{parskip.sty}{%
    \usepackage{parskip}
  }{% else
    \setlength{\parindent}{0pt}
    \setlength{\parskip}{6pt plus 2pt minus 1pt}}
}{% if KOMA class
  \KOMAoptions{parskip=half}}
\makeatother
\usepackage{xcolor}
\setlength{\emergencystretch}{3em} % prevent overfull lines
\setcounter{secnumdepth}{-\maxdimen} % remove section numbering
% Make \paragraph and \subparagraph free-standing
\makeatletter
\ifx\paragraph\undefined\else
  \let\oldparagraph\paragraph
  \renewcommand{\paragraph}{
    \@ifstar
      \xxxParagraphStar
      \xxxParagraphNoStar
  }
  \newcommand{\xxxParagraphStar}[1]{\oldparagraph*{#1}\mbox{}}
  \newcommand{\xxxParagraphNoStar}[1]{\oldparagraph{#1}\mbox{}}
\fi
\ifx\subparagraph\undefined\else
  \let\oldsubparagraph\subparagraph
  \renewcommand{\subparagraph}{
    \@ifstar
      \xxxSubParagraphStar
      \xxxSubParagraphNoStar
  }
  \newcommand{\xxxSubParagraphStar}[1]{\oldsubparagraph*{#1}\mbox{}}
  \newcommand{\xxxSubParagraphNoStar}[1]{\oldsubparagraph{#1}\mbox{}}
\fi
\makeatother

\usepackage{color}
\usepackage{fancyvrb}
\newcommand{\VerbBar}{|}
\newcommand{\VERB}{\Verb[commandchars=\\\{\}]}
\DefineVerbatimEnvironment{Highlighting}{Verbatim}{commandchars=\\\{\}}
% Add ',fontsize=\small' for more characters per line
\usepackage{framed}
\definecolor{shadecolor}{RGB}{241,243,245}
\newenvironment{Shaded}{\begin{snugshade}}{\end{snugshade}}
\newcommand{\AlertTok}[1]{\textcolor[rgb]{0.68,0.00,0.00}{#1}}
\newcommand{\AnnotationTok}[1]{\textcolor[rgb]{0.37,0.37,0.37}{#1}}
\newcommand{\AttributeTok}[1]{\textcolor[rgb]{0.40,0.45,0.13}{#1}}
\newcommand{\BaseNTok}[1]{\textcolor[rgb]{0.68,0.00,0.00}{#1}}
\newcommand{\BuiltInTok}[1]{\textcolor[rgb]{0.00,0.23,0.31}{#1}}
\newcommand{\CharTok}[1]{\textcolor[rgb]{0.13,0.47,0.30}{#1}}
\newcommand{\CommentTok}[1]{\textcolor[rgb]{0.37,0.37,0.37}{#1}}
\newcommand{\CommentVarTok}[1]{\textcolor[rgb]{0.37,0.37,0.37}{\textit{#1}}}
\newcommand{\ConstantTok}[1]{\textcolor[rgb]{0.56,0.35,0.01}{#1}}
\newcommand{\ControlFlowTok}[1]{\textcolor[rgb]{0.00,0.23,0.31}{\textbf{#1}}}
\newcommand{\DataTypeTok}[1]{\textcolor[rgb]{0.68,0.00,0.00}{#1}}
\newcommand{\DecValTok}[1]{\textcolor[rgb]{0.68,0.00,0.00}{#1}}
\newcommand{\DocumentationTok}[1]{\textcolor[rgb]{0.37,0.37,0.37}{\textit{#1}}}
\newcommand{\ErrorTok}[1]{\textcolor[rgb]{0.68,0.00,0.00}{#1}}
\newcommand{\ExtensionTok}[1]{\textcolor[rgb]{0.00,0.23,0.31}{#1}}
\newcommand{\FloatTok}[1]{\textcolor[rgb]{0.68,0.00,0.00}{#1}}
\newcommand{\FunctionTok}[1]{\textcolor[rgb]{0.28,0.35,0.67}{#1}}
\newcommand{\ImportTok}[1]{\textcolor[rgb]{0.00,0.46,0.62}{#1}}
\newcommand{\InformationTok}[1]{\textcolor[rgb]{0.37,0.37,0.37}{#1}}
\newcommand{\KeywordTok}[1]{\textcolor[rgb]{0.00,0.23,0.31}{\textbf{#1}}}
\newcommand{\NormalTok}[1]{\textcolor[rgb]{0.00,0.23,0.31}{#1}}
\newcommand{\OperatorTok}[1]{\textcolor[rgb]{0.37,0.37,0.37}{#1}}
\newcommand{\OtherTok}[1]{\textcolor[rgb]{0.00,0.23,0.31}{#1}}
\newcommand{\PreprocessorTok}[1]{\textcolor[rgb]{0.68,0.00,0.00}{#1}}
\newcommand{\RegionMarkerTok}[1]{\textcolor[rgb]{0.00,0.23,0.31}{#1}}
\newcommand{\SpecialCharTok}[1]{\textcolor[rgb]{0.37,0.37,0.37}{#1}}
\newcommand{\SpecialStringTok}[1]{\textcolor[rgb]{0.13,0.47,0.30}{#1}}
\newcommand{\StringTok}[1]{\textcolor[rgb]{0.13,0.47,0.30}{#1}}
\newcommand{\VariableTok}[1]{\textcolor[rgb]{0.07,0.07,0.07}{#1}}
\newcommand{\VerbatimStringTok}[1]{\textcolor[rgb]{0.13,0.47,0.30}{#1}}
\newcommand{\WarningTok}[1]{\textcolor[rgb]{0.37,0.37,0.37}{\textit{#1}}}

\providecommand{\tightlist}{%
  \setlength{\itemsep}{0pt}\setlength{\parskip}{0pt}}\usepackage{longtable,booktabs,array}
\usepackage{calc} % for calculating minipage widths
% Correct order of tables after \paragraph or \subparagraph
\usepackage{etoolbox}
\makeatletter
\patchcmd\longtable{\par}{\if@noskipsec\mbox{}\fi\par}{}{}
\makeatother
% Allow footnotes in longtable head/foot
\IfFileExists{footnotehyper.sty}{\usepackage{footnotehyper}}{\usepackage{footnote}}
\makesavenoteenv{longtable}
\usepackage{graphicx}
\makeatletter
\def\maxwidth{\ifdim\Gin@nat@width>\linewidth\linewidth\else\Gin@nat@width\fi}
\def\maxheight{\ifdim\Gin@nat@height>\textheight\textheight\else\Gin@nat@height\fi}
\makeatother
% Scale images if necessary, so that they will not overflow the page
% margins by default, and it is still possible to overwrite the defaults
% using explicit options in \includegraphics[width, height, ...]{}
\setkeys{Gin}{width=\maxwidth,height=\maxheight,keepaspectratio}
% Set default figure placement to htbp
\makeatletter
\def\fps@figure{htbp}
\makeatother
% definitions for citeproc citations
\NewDocumentCommand\citeproctext{}{}
\NewDocumentCommand\citeproc{mm}{%
  \begingroup\def\citeproctext{#2}\cite{#1}\endgroup}
\makeatletter
 % allow citations to break across lines
 \let\@cite@ofmt\@firstofone
 % avoid brackets around text for \cite:
 \def\@biblabel#1{}
 \def\@cite#1#2{{#1\if@tempswa , #2\fi}}
\makeatother
\newlength{\cslhangindent}
\setlength{\cslhangindent}{1.5em}
\newlength{\csllabelwidth}
\setlength{\csllabelwidth}{3em}
\newenvironment{CSLReferences}[2] % #1 hanging-indent, #2 entry-spacing
 {\begin{list}{}{%
  \setlength{\itemindent}{0pt}
  \setlength{\leftmargin}{0pt}
  \setlength{\parsep}{0pt}
  % turn on hanging indent if param 1 is 1
  \ifodd #1
   \setlength{\leftmargin}{\cslhangindent}
   \setlength{\itemindent}{-1\cslhangindent}
  \fi
  % set entry spacing
  \setlength{\itemsep}{#2\baselineskip}}}
 {\end{list}}
\usepackage{calc}
\newcommand{\CSLBlock}[1]{\hfill\break\parbox[t]{\linewidth}{\strut\ignorespaces#1\strut}}
\newcommand{\CSLLeftMargin}[1]{\parbox[t]{\csllabelwidth}{\strut#1\strut}}
\newcommand{\CSLRightInline}[1]{\parbox[t]{\linewidth - \csllabelwidth}{\strut#1\strut}}
\newcommand{\CSLIndent}[1]{\hspace{\cslhangindent}#1}

\KOMAoption{captions}{tableheading}
\makeatletter
\@ifpackageloaded{caption}{}{\usepackage{caption}}
\AtBeginDocument{%
\ifdefined\contentsname
  \renewcommand*\contentsname{Table of contents}
\else
  \newcommand\contentsname{Table of contents}
\fi
\ifdefined\listfigurename
  \renewcommand*\listfigurename{List of Figures}
\else
  \newcommand\listfigurename{List of Figures}
\fi
\ifdefined\listtablename
  \renewcommand*\listtablename{List of Tables}
\else
  \newcommand\listtablename{List of Tables}
\fi
\ifdefined\figurename
  \renewcommand*\figurename{Figure}
\else
  \newcommand\figurename{Figure}
\fi
\ifdefined\tablename
  \renewcommand*\tablename{Table}
\else
  \newcommand\tablename{Table}
\fi
}
\@ifpackageloaded{float}{}{\usepackage{float}}
\floatstyle{ruled}
\@ifundefined{c@chapter}{\newfloat{codelisting}{h}{lop}}{\newfloat{codelisting}{h}{lop}[chapter]}
\floatname{codelisting}{Listing}
\newcommand*\listoflistings{\listof{codelisting}{List of Listings}}
\makeatother
\makeatletter
\makeatother
\makeatletter
\@ifpackageloaded{caption}{}{\usepackage{caption}}
\@ifpackageloaded{subcaption}{}{\usepackage{subcaption}}
\makeatother

\ifLuaTeX
  \usepackage{selnolig}  % disable illegal ligatures
\fi
\usepackage{bookmark}

\IfFileExists{xurl.sty}{\usepackage{xurl}}{} % add URL line breaks if available
\urlstyle{same} % disable monospaced font for URLs
\hypersetup{
  pdftitle={Shifts in Values, Shifts in Democracy: Modeling and Preliminary Discussion},
  pdfauthor={Jason Gordon - jagordon@uchicago.edu},
  colorlinks=true,
  linkcolor={blue},
  filecolor={Maroon},
  citecolor={Blue},
  urlcolor={Blue},
  pdfcreator={LaTeX via pandoc}}


\title{Shifts in Values, Shifts in Democracy: Modeling and Preliminary
Discussion}
\author{Jason Gordon - jagordon@uchicago.edu}
\date{2025-02-24}

\begin{document}
\maketitle


\newpage
\tableofcontents
\newpage

\subsection{Intoduction}\label{intoduction}

Democratic backsliding has increasingly taken hold of the world, with
the Varieties of Democracy (V-DEM) 2024 Democracy Report identifying 28
cases of backsliding in democracies, 13 of which having reverted to
autocracies since the episode began.\footnote{Marina Nord et al.,
  {``Democracy {Report} 2024: {Democracy Winning} and {Losing} at the
  {Ballot}''} (University of Gothenburg: V-Dem Institute, March 2024).}
As backsliding occurs within democracies with elected leaders, there is
seemingly popular support \emph{against} democracy in these countries
--- at least given the results of these elections.

In this report, I plan to inspect if the wave of autocratization in
democracies is the result of a shift in values against democracy, and
which values are the most important for the shift to autocracy. As the
number of cases of backsliding increases, it becomes increasingly
important to understand why people would vote for actors that seek to
challenge the core foundation of democratic institutions; with this
knowledge, we may be able to further understand when cases of
backsliding can happen.

First, I will be summarizing existing literature on democratic
backsliding, identifying the tension in what causes backsliding. Then, I
will conduct some exploratory data analysis and model fitting of my
dataset, using Lasso to identify notable variables. Finally, I will
discuss if these variables have importance in conducting future
research.

\subsection{Existing Literature}\label{existing-literature}

Over the past twenty-five years, there has been considerable discussion
about the causes of democratic backsliding - defined generally as the
gradual breakdown of democratic institutions through nonviolent --- and
often, legal --- means.\footnote{Stephan Haggard and Robert Kaufman,
  \emph{Backsliding: {Democratic Regress} in the {Contemporary World}},
  Elements in {Political Economy} 9 (Cambridge University Press, 2021);
  Steven Levitsky and Daniel Ziblatt, \emph{How {Democracies Die}},
  First edition (New York: Crown, 2018).} Most discussions claim
polarization is the core idea behind why democracies denigrate, but
little is known about the shifts in values to \emph{why} it happens.

Haggard and Kaufman (2021a) discuss backsliding through the lens of
dysfunction of a lack of trust in institutions, which permit autocrats
to capture the executive.\footnote{Stephan Haggard and Robert Kaufman,
  {``The {Anatomy} of {Democratic Backsliding},''} \emph{Journal of
  Democracy} 32, no. 4 (2021): 27--41.} These scholars point to an
overall frustration with the system leading to polarization, as
anti-system actors may be able to succeed in taking power if the
population has no faith that the system will perform well with them.
Other literature seems to correlate parts of this claim, with Cooley and
Nexon (2020) arguing that these are akin to a ``counter-order''
movement, made in rejection to the liberal system of
ordering.\footnote{Alexander Cooley and Daniel Nexon, \emph{Exit from
  {Hegemony}: {The Unraveling} of the {American Global Order}} (Oxford,
  New York: Oxford University Press, 2020).} Generally, these scholars
agree that movements and politicians rejecting the structure of
governance as it stands currently obtain success, with Haggard and
Kaufman finding it is due to dissatisfaction in institutions.

However, Carothers and Press (2022) expand on this discussion, first
limiting their discussion to the Global South and post-communist Europe,
and delivering three separate categories of democratic backsliding, all
of which create differing interpretations of the values behind
backsliding. They state that backsliding can be caused by
grievance-fueled illiberalism, agreeing with future literature that
frustrations in institutions can drive backsliding; opportunistic
authoritarianism, where autocrats act as political entrepreneurs to
obtain power; and entrenched-interest revanchinism, where an interest
group uses undemocratic means to reassert its claim.\footnote{Thomas
  Carothers and Benjamin Press, {``Understanding and {Responding} to
  {Global Democratic Backsliding}''} (Carnegie Endowment for
  International Peace, October 2022).} Carothers and Press's framework
gives an idea for the potential variance in the viewpoints behind
backsliding, claiming it is not as simple as people being mad about
institutions failing them.

An additional perspective claims that backsliding isn't domestic in
nature, but international. One such scholar, Anna Meyerrose (2020;
2021), finds that instead of backsliding being triggered by domestic
parties, that international organizations (IOs) are to blame. She claims
that because IOs focus on elections and an increase in executive power
in their democracy promotion, in addition to removing some areas of
policy for political parties to divide on, potential autocrats have an
easier time gaining and maintaining power.\footnote{Anna M. Meyerrose,
  {``International {Sources} of {Democratic Backsliding},''} in
  \emph{Routledge {Handbook} of {Illiberalism}} (Routledge, 2021); Anna
  M. Meyerrose, {``The {Unintended Consequences} of {Democracy
  Promotion}: {International Organizations} and {Democratic
  Backsliding},''} \emph{Comparative Political Studies} 53, no. 10-11
  (September 2020): 1547--81,
  \url{https://doi.org/10.1177/0010414019897689}.} Deviating from other
arguments, this branch of literature finds that while there may be
domestic issues, structural problems make backsliding \emph{much} more
likely. As a result, backsliding may have nothing to do with values
populations have.

I intend to put these differing claims to the test. While I expect that
grievances with institutions play a not insignificant role in democratic
backsliding, I do expect there to be variance in how much of a role it
plays. Some states may backslide due to structural issues, or for
electing a leader who at first may adhere to democratic norms, but then
turn against the system later in their term.

\subsection{Data}\label{data}

I plan to combine the data from V-DEM's 2024 Democracy Report and the
World Values Survey to identify correlations WVS's answers and overall
level of democracy. I'm looking to use waves 6 and 7 of the WVS report,
so will only need V-DEM's data beyond 2010.

\begin{Shaded}
\begin{Highlighting}[]
\FunctionTok{library}\NormalTok{(tidyverse)}
\end{Highlighting}
\end{Shaded}

\begin{verbatim}
-- Attaching core tidyverse packages ------------------------ tidyverse 2.0.0 --
v dplyr     1.1.4     v readr     2.1.5
v forcats   1.0.0     v stringr   1.5.1
v ggplot2   3.5.1     v tibble    3.2.1
v lubridate 1.9.3     v tidyr     1.3.1
v purrr     1.0.2     
-- Conflicts ------------------------------------------ tidyverse_conflicts() --
x dplyr::filter() masks stats::filter()
x dplyr::lag()    masks stats::lag()
i Use the conflicted package (<http://conflicted.r-lib.org/>) to force all conflicts to become errors
\end{verbatim}

\begin{Shaded}
\begin{Highlighting}[]
\FunctionTok{library}\NormalTok{(vdemdata)}
\FunctionTok{library}\NormalTok{(glmnet)}
\end{Highlighting}
\end{Shaded}

\begin{verbatim}
Warning: package 'glmnet' was built under R version 4.4.2
\end{verbatim}

\begin{verbatim}
Loading required package: Matrix

Attaching package: 'Matrix'

The following objects are masked from 'package:tidyr':

    expand, pack, unpack

Loaded glmnet 4.1-8
\end{verbatim}

\begin{Shaded}
\begin{Highlighting}[]
\FunctionTok{load}\NormalTok{(}\StringTok{\textquotesingle{}WVS\_Cross{-}National\_Wave\_7\_rData\_v6\_0.rdata\textquotesingle{}}\NormalTok{)}
\NormalTok{wvs7 }\OtherTok{\textless{}{-}} \StringTok{\textasciigrave{}}\AttributeTok{WVS\_Cross{-}National\_Wave\_7\_v6\_0}\StringTok{\textasciigrave{}} \SpecialCharTok{|\textgreater{}} \FunctionTok{as.tibble}\NormalTok{()}
\end{Highlighting}
\end{Shaded}

\begin{verbatim}
Warning: `as.tibble()` was deprecated in tibble 2.0.0.
i Please use `as_tibble()` instead.
i The signature and semantics have changed, see `?as_tibble`.
\end{verbatim}

\begin{Shaded}
\begin{Highlighting}[]
\FunctionTok{load}\NormalTok{(}\StringTok{\textquotesingle{}WV6\_Data\_R\_v20201117.rdata\textquotesingle{}}\NormalTok{)}
\NormalTok{wvs6 }\OtherTok{\textless{}{-}} \StringTok{\textasciigrave{}}\AttributeTok{WV6\_Data\_R\_v20201117}\StringTok{\textasciigrave{}} \SpecialCharTok{|\textgreater{}} \FunctionTok{as.tibble}\NormalTok{()}

\NormalTok{workingvdem }\OtherTok{\textless{}{-}} \FunctionTok{tibble}\NormalTok{(vdem) }\SpecialCharTok{|\textgreater{}}
  \FunctionTok{select}\NormalTok{(}\FunctionTok{c}\NormalTok{(country\_name, country\_text\_id, year, v2x\_polyarchy, v2x\_regime\_amb)) }\SpecialCharTok{|\textgreater{}}
  \FunctionTok{filter}\NormalTok{(year }\SpecialCharTok{\textgreater{}=} \DecValTok{2005}\NormalTok{)}
\end{Highlighting}
\end{Shaded}

First, I will clean the V-DEM dataset, adding a difference in the
\texttt{v2x\_polyarchy} variable - that is, the measure of electoral
democracy - and create a boolean variable for if the state has
experienced a backsliding episode, defined as having its electoral
democracy score decrease at least three years in a row. I will also
ensure any NAs introduced are properly handled at this stage; since my
focus is after 2010, I am removing the years before then, but will
filter for year after the variables are introduced to limit NAs. The
remaining are due to South Sudan not existing before 2011 and NAs
introduced by attempting to get data after 2023, when the V-DEM dataset
ends.

\begin{Shaded}
\begin{Highlighting}[]
\NormalTok{workingvdem }\OtherTok{\textless{}{-}}\NormalTok{ workingvdem }\SpecialCharTok{|\textgreater{}}
  \FunctionTok{group\_by}\NormalTok{(country\_text\_id) }\SpecialCharTok{|\textgreater{}}
  \FunctionTok{arrange}\NormalTok{(year) }\SpecialCharTok{|\textgreater{}} \CommentTok{\# Should already be the case, but just in case it\textquotesingle{}s not}
  \FunctionTok{mutate}\NormalTok{(}\AttributeTok{diff\_polyarchy =}\NormalTok{ v2x\_polyarchy }\SpecialCharTok{{-}} \FunctionTok{lag}\NormalTok{(v2x\_polyarchy)) }\SpecialCharTok{|\textgreater{}}
  \FunctionTok{ungroup}\NormalTok{() }\SpecialCharTok{|\textgreater{}}
  \FunctionTok{arrange}\NormalTok{(country\_text\_id)}

\NormalTok{workingvdem }\OtherTok{\textless{}{-}}\NormalTok{ workingvdem }\SpecialCharTok{|\textgreater{}} \FunctionTok{group\_by}\NormalTok{(country\_text\_id) }\SpecialCharTok{|\textgreater{}}
  \CommentTok{\# Did electoral democracy decrease in a non{-}entrenched autocracy?}
  \FunctionTok{mutate}\NormalTok{(}\AttributeTok{backslided =}\NormalTok{ (diff\_polyarchy }\SpecialCharTok{\textless{}} \SpecialCharTok{{-}}\FloatTok{0.01}\NormalTok{) }\SpecialCharTok{\&}\NormalTok{ (v2x\_regime\_amb) }\SpecialCharTok{\textgreater{}} \DecValTok{3}\NormalTok{) }\SpecialCharTok{|\textgreater{}}
  \CommentTok{\# Did it do it at least three years in a row?}
  \FunctionTok{mutate}\NormalTok{(}\AttributeTok{backslided =}\NormalTok{ (backslided }\SpecialCharTok{\&} \FunctionTok{lag}\NormalTok{(backslided) }\SpecialCharTok{\&} \FunctionTok{lag}\NormalTok{(}\FunctionTok{lag}\NormalTok{(backslided))) }\SpecialCharTok{|}
           \FunctionTok{lead}\NormalTok{(backslided) }\SpecialCharTok{\&}\NormalTok{ backslided }\SpecialCharTok{\&} \FunctionTok{lag}\NormalTok{(backslided) }\SpecialCharTok{|}
           \FunctionTok{lead}\NormalTok{(}\FunctionTok{lead}\NormalTok{(backslided)) }\SpecialCharTok{\&} \FunctionTok{lead}\NormalTok{(backslided) }\SpecialCharTok{\&}\NormalTok{ backslided) }\SpecialCharTok{|\textgreater{}}
  \FunctionTok{filter}\NormalTok{(year }\SpecialCharTok{\textgreater{}} \DecValTok{2009}\NormalTok{)}
\CommentTok{\# 2022/2023 values (is NA for ones in backsliding. Let\textquotesingle{}s make that True.)}
\NormalTok{workingvdem}\SpecialCharTok{$}\NormalTok{backslided }\OtherTok{\textless{}{-}} \FunctionTok{ifelse}\NormalTok{(}\FunctionTok{is.na}\NormalTok{(workingvdem}\SpecialCharTok{$}\NormalTok{backslided),}
                                 \ConstantTok{TRUE}\NormalTok{,}
\NormalTok{                                 workingvdem}\SpecialCharTok{$}\NormalTok{backslided)}

\NormalTok{isnum\_lowna }\OtherTok{\textless{}{-}} \FunctionTok{apply}\NormalTok{(workingvdem, }\DecValTok{2}\NormalTok{, }\ControlFlowTok{function}\NormalTok{(col)\{}
  \FunctionTok{sum}\NormalTok{(}\FunctionTok{is.na}\NormalTok{(col))}
\NormalTok{\})}
\CommentTok{\# The 1 is for South Sudan here.}
\NormalTok{na\_rows }\OtherTok{\textless{}{-}} \FunctionTok{apply}\NormalTok{(workingvdem, }\DecValTok{1}\NormalTok{, }\ControlFlowTok{function}\NormalTok{(row)\{}\FunctionTok{sum}\NormalTok{(}\FunctionTok{is.na}\NormalTok{(row))\})}
\NormalTok{missing\_rows }\OtherTok{\textless{}{-}}\NormalTok{ na\_rows }\SpecialCharTok{!=} \DecValTok{0}
\NormalTok{workingvdem }\OtherTok{\textless{}{-}}\NormalTok{ workingvdem[}\SpecialCharTok{!}\NormalTok{missing\_rows,]}
\end{Highlighting}
\end{Shaded}

I will do a similar level of cleaning for the two WVS surveys I have,
but will work on them separately due to different questions asked in
each wave. I will summarize answers by country to get a single value to
compare to as well. To take care of negative values, which are NA in
this case, I have attempted to remove them through taking the averages
of a country's data and then set any remaining NA values to -1.
Otherwise, the data tends to get messy. This may complicate the data,
but ideally in a somewhat uniform fashion.

\begin{Shaded}
\begin{Highlighting}[]
\NormalTok{wvs\_test }\OtherTok{\textless{}{-}}\NormalTok{ dplyr}\SpecialCharTok{::}\FunctionTok{select}\NormalTok{(wvs7, A\_YEAR, B\_COUNTRY\_ALPHA, }\FunctionTok{matches}\NormalTok{(}\StringTok{"\^{}Q([0{-}9]+)"}\NormalTok{)) }\SpecialCharTok{|\textgreater{}}
  \FunctionTok{rename}\NormalTok{(}\AttributeTok{year =}\NormalTok{ A\_YEAR, }\AttributeTok{country\_text\_id =}\NormalTok{ B\_COUNTRY\_ALPHA)}

\NormalTok{isnum }\OtherTok{\textless{}{-}} \FunctionTok{sapply}\NormalTok{(wvs\_test, is.numeric) }\CommentTok{\# Cols where we\textquotesingle{}re working with \#\textquotesingle{}s}
\CommentTok{\# For those cols, remove all vals less than 0.}
\NormalTok{wvs\_test[isnum] }\OtherTok{\textless{}{-}} \FunctionTok{lapply}\NormalTok{(wvs\_test[isnum], \textbackslash{}(x) }\FunctionTok{ifelse}\NormalTok{(x }\SpecialCharTok{\textless{}} \DecValTok{0}\NormalTok{, }\ConstantTok{NA}\NormalTok{, x))}

\CommentTok{\# By country}
\NormalTok{wvs\_test }\OtherTok{\textless{}{-}}\NormalTok{ wvs\_test }\SpecialCharTok{|\textgreater{}}
  \FunctionTok{group\_by}\NormalTok{(year, country\_text\_id) }\SpecialCharTok{|\textgreater{}}
  \FunctionTok{summarize}\NormalTok{(}\FunctionTok{across}\NormalTok{(}\FunctionTok{everything}\NormalTok{(), \textbackslash{}(x) }\FunctionTok{mean}\NormalTok{(x, }\AttributeTok{na.rm =} \ConstantTok{TRUE}\NormalTok{)))}
\end{Highlighting}
\end{Shaded}

\begin{verbatim}
`summarise()` has grouped output by 'year'. You can override using the
`.groups` argument.
\end{verbatim}

\begin{Shaded}
\begin{Highlighting}[]
\NormalTok{isnum }\OtherTok{\textless{}{-}} \FunctionTok{sapply}\NormalTok{(wvs\_test, is.numeric) }\CommentTok{\# Cols where we\textquotesingle{}re working with \#\textquotesingle{}s}
\CommentTok{\# For those cols, set all remaining NAs to {-}1.}
\NormalTok{wvs\_test[isnum] }\OtherTok{\textless{}{-}} \FunctionTok{lapply}\NormalTok{(wvs\_test[isnum], \textbackslash{}(x) }\FunctionTok{ifelse}\NormalTok{(}\FunctionTok{is.na}\NormalTok{(x), }\SpecialCharTok{{-}}\DecValTok{1}\NormalTok{, x))}

\NormalTok{merged }\OtherTok{\textless{}{-}} \FunctionTok{merge}\NormalTok{(workingvdem, wvs\_test, }\AttributeTok{by =} \FunctionTok{c}\NormalTok{(}\StringTok{"year"}\NormalTok{, }\StringTok{"country\_text\_id"}\NormalTok{)) }\SpecialCharTok{|\textgreater{}}
  \FunctionTok{as\_tibble}\NormalTok{()}
\end{Highlighting}
\end{Shaded}

To determine which features are the most important, I plan on utilizing
a Lasso model. I initially used \texttt{diff\_polyarchy} for my metric
to determine key features on and scaled it so its variance was
significant, which allowed me to use lasso, but it does not allow me to
get the top features since the variance is still rather low.

Instead, I used the electoral democracy score (\texttt{v2x\_polyarchy}),
which functioned a lot better, allowing me to find the top features and
plotting things properly.

\begin{Shaded}
\begin{Highlighting}[]
\NormalTok{grid }\OtherTok{\textless{}{-}} \DecValTok{10}\SpecialCharTok{\^{}}\FunctionTok{seq}\NormalTok{(}\DecValTok{10}\NormalTok{, }\SpecialCharTok{{-}}\DecValTok{2}\NormalTok{, }\AttributeTok{length =} \DecValTok{100}\NormalTok{)}
\NormalTok{y7 }\OtherTok{\textless{}{-}}\NormalTok{ merged}\SpecialCharTok{$}\NormalTok{v2x\_polyarchy}
\NormalTok{z7 }\OtherTok{\textless{}{-}}\NormalTok{ merged}\SpecialCharTok{$}\NormalTok{diff\_polyarchy }\SpecialCharTok{|\textgreater{}} \FunctionTok{scale}\NormalTok{()}
\NormalTok{merged\_lasso }\OtherTok{\textless{}{-}}\NormalTok{ dplyr}\SpecialCharTok{::}\FunctionTok{select}\NormalTok{(merged, }\FunctionTok{matches}\NormalTok{(}\StringTok{"\^{}Q([0{-}9]+)"}\NormalTok{))}

\NormalTok{lasso\_mod }\OtherTok{\textless{}{-}} \FunctionTok{glmnet}\NormalTok{(}\FunctionTok{data.matrix}\NormalTok{(merged\_lasso), y7, }\AttributeTok{alpha =} \DecValTok{1}\NormalTok{, }\AttributeTok{lambda =}\NormalTok{ grid)}
\FunctionTok{plot}\NormalTok{(lasso\_mod)}
\end{Highlighting}
\end{Shaded}

\begin{verbatim}
Warning in regularize.values(x, y, ties, missing(ties), na.rm = na.rm):
collapsing to unique 'x' values
\end{verbatim}

\includegraphics{Episode_2_files/figure-pdf/lasso-1.pdf}

\begin{Shaded}
\begin{Highlighting}[]
\FunctionTok{set.seed}\NormalTok{(}\DecValTok{1}\NormalTok{)}
\NormalTok{cv7\_out }\OtherTok{\textless{}{-}} \FunctionTok{cv.glmnet}\NormalTok{(}\FunctionTok{data.matrix}\NormalTok{(merged\_lasso), y7, }\AttributeTok{alpha =} \DecValTok{1}\NormalTok{)}
\FunctionTok{plot}\NormalTok{(cv7\_out)}
\end{Highlighting}
\end{Shaded}

\includegraphics{Episode_2_files/figure-pdf/lasso-2.pdf}

\begin{Shaded}
\begin{Highlighting}[]
\NormalTok{top\_features7 }\OtherTok{\textless{}{-}} \FunctionTok{coef}\NormalTok{(cv7\_out, }\AttributeTok{s =} \StringTok{"lambda.1se"}\NormalTok{)}
\NormalTok{top\_features7 }\OtherTok{\textless{}{-}}\NormalTok{ top\_features7[}\SpecialCharTok{{-}}\DecValTok{1}\NormalTok{,]}
\NormalTok{top\_features7 }\OtherTok{\textless{}{-}}\NormalTok{ top\_features7[top\_features7 }\SpecialCharTok{!=} \DecValTok{0}\NormalTok{]}
\FunctionTok{sort}\NormalTok{(}\FunctionTok{abs}\NormalTok{(top\_features7), }\AttributeTok{decreasing =} \ConstantTok{TRUE}\NormalTok{) }\SpecialCharTok{|\textgreater{}} \FunctionTok{head}\NormalTok{(}\DecValTok{15}\NormalTok{)}
\end{Highlighting}
\end{Shaded}

\begin{verbatim}
        Q211         Q239          Q38         Q212         Q271         Q225 
0.1967821941 0.0559778577 0.0482793422 0.0480802239 0.0364528223 0.0228602491 
         Q73           Q3        Q292I         Q182         Q262 
0.0190355263 0.0151253037 0.0107729186 0.0044779803 0.0001399158 
\end{verbatim}

\begin{Shaded}
\begin{Highlighting}[]
\NormalTok{lasso\_modz }\OtherTok{\textless{}{-}} \FunctionTok{glmnet}\NormalTok{(}\FunctionTok{data.matrix}\NormalTok{(merged\_lasso), z7, }\AttributeTok{alpha =} \DecValTok{1}\NormalTok{, }\AttributeTok{lambda =}\NormalTok{ grid)}
\FunctionTok{plot}\NormalTok{(lasso\_modz)}
\end{Highlighting}
\end{Shaded}

\begin{verbatim}
Warning in regularize.values(x, y, ties, missing(ties), na.rm = na.rm):
collapsing to unique 'x' values
\end{verbatim}

\includegraphics{Episode_2_files/figure-pdf/lasso-3.pdf}

\begin{Shaded}
\begin{Highlighting}[]
\CommentTok{\# However, can\textquotesingle{}t find most important features because everything\textquotesingle{}s so small!}
\end{Highlighting}
\end{Shaded}

We can do the same process for the 6th wave, as seen below.

\begin{Shaded}
\begin{Highlighting}[]
\NormalTok{wvs6\_test }\OtherTok{\textless{}{-}}\NormalTok{ dplyr}\SpecialCharTok{::}\FunctionTok{select}\NormalTok{(wvs6, B\_COUNTRY\_ALPHA, }\FunctionTok{matches}\NormalTok{(}\StringTok{"\^{}V([0{-}9]+)"}\NormalTok{)) }\SpecialCharTok{|\textgreater{}}
\NormalTok{  dplyr}\SpecialCharTok{::}\FunctionTok{select}\NormalTok{(}\SpecialCharTok{{-}}\FunctionTok{c}\NormalTok{(V1, V2, V2A, V3)) }\SpecialCharTok{|\textgreater{}}
  \FunctionTok{rename}\NormalTok{(}\AttributeTok{country\_text\_id =}\NormalTok{ B\_COUNTRY\_ALPHA)}

\NormalTok{wvs6\_test}\SpecialCharTok{$}\NormalTok{year }\OtherTok{=} \DecValTok{2013} \CommentTok{\# no specific year, assuming 2013.}

\NormalTok{isnum }\OtherTok{\textless{}{-}} \FunctionTok{sapply}\NormalTok{(wvs6\_test, is.numeric) }\CommentTok{\# Cols where we\textquotesingle{}re working with \#\textquotesingle{}s}
\CommentTok{\# For those cols, remove all vals less than 0.}
\NormalTok{wvs6\_test[isnum] }\OtherTok{\textless{}{-}} \FunctionTok{lapply}\NormalTok{(wvs6\_test[isnum], \textbackslash{}(x) }\FunctionTok{ifelse}\NormalTok{(x }\SpecialCharTok{\textless{}} \DecValTok{0}\NormalTok{, }\ConstantTok{NA}\NormalTok{, x))}

\CommentTok{\# By country}
\NormalTok{wvs6\_test }\OtherTok{\textless{}{-}}\NormalTok{ wvs6\_test }\SpecialCharTok{|\textgreater{}}
  \FunctionTok{group\_by}\NormalTok{(year, country\_text\_id) }\SpecialCharTok{|\textgreater{}}
  \FunctionTok{summarize}\NormalTok{(}\FunctionTok{across}\NormalTok{(}\FunctionTok{everything}\NormalTok{(), \textbackslash{}(x) }\FunctionTok{mean}\NormalTok{(x, }\AttributeTok{na.rm =} \ConstantTok{TRUE}\NormalTok{)))}
\end{Highlighting}
\end{Shaded}

\begin{verbatim}
`summarise()` has grouped output by 'year'. You can override using the
`.groups` argument.
\end{verbatim}

\begin{Shaded}
\begin{Highlighting}[]
\NormalTok{isnum }\OtherTok{\textless{}{-}} \FunctionTok{sapply}\NormalTok{(wvs6\_test, is.numeric) }\CommentTok{\# Cols where we\textquotesingle{}re working with \#\textquotesingle{}s}
\CommentTok{\# For those cols, remove all vals less than 0.}
\NormalTok{wvs6\_test[isnum] }\OtherTok{\textless{}{-}} \FunctionTok{lapply}\NormalTok{(wvs6\_test[isnum], \textbackslash{}(x) }\FunctionTok{ifelse}\NormalTok{(}\FunctionTok{is.na}\NormalTok{(x), }\SpecialCharTok{{-}}\DecValTok{1}\NormalTok{, x))}

\NormalTok{merged6 }\OtherTok{\textless{}{-}} \FunctionTok{merge}\NormalTok{(workingvdem, wvs6\_test, }\AttributeTok{by =} \FunctionTok{c}\NormalTok{(}\StringTok{"year"}\NormalTok{, }\StringTok{"country\_text\_id"}\NormalTok{)) }\SpecialCharTok{|\textgreater{}}
  \FunctionTok{as\_tibble}\NormalTok{()}

\NormalTok{y6 }\OtherTok{\textless{}{-}}\NormalTok{ merged6}\SpecialCharTok{$}\NormalTok{v2x\_polyarchy}
\NormalTok{z6 }\OtherTok{\textless{}{-}}\NormalTok{ merged6}\SpecialCharTok{$}\NormalTok{diff\_polyarchy }\SpecialCharTok{|\textgreater{}} \FunctionTok{scale}\NormalTok{()}
\NormalTok{merged6\_lasso }\OtherTok{\textless{}{-}}\NormalTok{ dplyr}\SpecialCharTok{::}\FunctionTok{select}\NormalTok{(merged6, }\FunctionTok{matches}\NormalTok{(}\StringTok{"\^{}V([0{-}9]+)"}\NormalTok{)) }\SpecialCharTok{|\textgreater{}}
\NormalTok{  dplyr}\SpecialCharTok{::}\FunctionTok{select}\NormalTok{(}\SpecialCharTok{{-}}\FunctionTok{matches}\NormalTok{(}\StringTok{"\^{}v2x"}\NormalTok{)) }\CommentTok{\# Have to EXPLICITLY remove these.}

\NormalTok{lasso\_modz }\OtherTok{\textless{}{-}} \FunctionTok{glmnet}\NormalTok{(}\FunctionTok{data.matrix}\NormalTok{(merged6\_lasso), z6, }\AttributeTok{alpha =} \DecValTok{1}\NormalTok{, }\AttributeTok{lambda =}\NormalTok{ grid)}
\FunctionTok{plot}\NormalTok{(lasso\_modz)}
\end{Highlighting}
\end{Shaded}

\begin{verbatim}
Warning in regularize.values(x, y, ties, missing(ties), na.rm = na.rm):
collapsing to unique 'x' values
\end{verbatim}

\includegraphics{Episode_2_files/figure-pdf/wvs6-1.pdf}

\begin{Shaded}
\begin{Highlighting}[]
\CommentTok{\# However, can\textquotesingle{}t find most important features because everything\textquotesingle{}s so small!}

\NormalTok{lasso\_mod6 }\OtherTok{\textless{}{-}} \FunctionTok{glmnet}\NormalTok{(}\FunctionTok{data.matrix}\NormalTok{(merged6\_lasso), y6, }\AttributeTok{alpha =} \DecValTok{1}\NormalTok{, }\AttributeTok{lambda =}\NormalTok{ grid)}
\FunctionTok{plot}\NormalTok{(lasso\_mod6)}
\end{Highlighting}
\end{Shaded}

\begin{verbatim}
Warning in regularize.values(x, y, ties, missing(ties), na.rm = na.rm):
collapsing to unique 'x' values
\end{verbatim}

\includegraphics{Episode_2_files/figure-pdf/wvs6-2.pdf}

\begin{Shaded}
\begin{Highlighting}[]
\FunctionTok{set.seed}\NormalTok{(}\DecValTok{1}\NormalTok{)}
\NormalTok{cv6\_out }\OtherTok{\textless{}{-}} \FunctionTok{cv.glmnet}\NormalTok{(}\FunctionTok{data.matrix}\NormalTok{(merged6\_lasso), y6, }\AttributeTok{alpha =} \DecValTok{1}\NormalTok{)}
\FunctionTok{plot}\NormalTok{(cv6\_out)}
\end{Highlighting}
\end{Shaded}

\includegraphics{Episode_2_files/figure-pdf/wvs6-3.pdf}

\begin{Shaded}
\begin{Highlighting}[]
\NormalTok{top\_features6 }\OtherTok{\textless{}{-}} \FunctionTok{coef}\NormalTok{(cv6\_out, }\AttributeTok{s =} \StringTok{"lambda.1se"}\NormalTok{)}
\NormalTok{top\_features6 }\OtherTok{\textless{}{-}}\NormalTok{ top\_features6[}\SpecialCharTok{{-}}\DecValTok{1}\NormalTok{,]}
\NormalTok{top\_features6 }\OtherTok{\textless{}{-}}\NormalTok{ top\_features6[top\_features6 }\SpecialCharTok{!=} \DecValTok{0}\NormalTok{]}
\FunctionTok{sort}\NormalTok{(}\FunctionTok{abs}\NormalTok{(top\_features6), }\AttributeTok{decreasing =} \ConstantTok{TRUE}\NormalTok{) }\SpecialCharTok{|\textgreater{}} \FunctionTok{head}\NormalTok{(}\DecValTok{15}\NormalTok{)}
\end{Highlighting}
\end{Shaded}

\begin{verbatim}
       V115         V51         V52        V203          V6         V53 
0.100867951 0.077406804 0.054297656 0.030849359 0.030637348 0.015164859 
       V208     V215_11 
0.009721931 0.002571395 
\end{verbatim}

\subsection{Discussion}\label{discussion}

As these values give differing results, I'll be analyzing them
separately before discussing them in tandem.

\subsubsection{Wave 6}\label{wave-6}

The top correlated variables with electoral democracy level after the
cross-validation on WVS 6th wave are as follows: V115 \emph{(confidence
in government)}, V51 \emph{(Do men make better political leaders?)}, V52
\emph{(Is a university education more important for men over women?)},
V203 \emph{(Is homosexuality justifiable?)}, V6 \emph{(Importance of
leisure time)}, V53 \emph{(Do men make better business executives?)},
V208 \emph{(Is it justifiable for a man to beat his wife?)}, and
V215\_11 \emph{(Seeing oneself as part of CIS)}. You can also see the
graph with them labeled below.

\begin{Shaded}
\begin{Highlighting}[]
\NormalTok{merged6\_select }\OtherTok{\textless{}{-}}\NormalTok{ merged6\_lasso }\SpecialCharTok{|\textgreater{}}\NormalTok{ dplyr}\SpecialCharTok{::}\FunctionTok{select}\NormalTok{(}\FunctionTok{names}\NormalTok{(top\_features6))}
\NormalTok{lasso\_m6s }\OtherTok{\textless{}{-}} \FunctionTok{glmnet}\NormalTok{(}\FunctionTok{data.matrix}\NormalTok{(merged6\_select), y6, }\AttributeTok{alpha =} \DecValTok{1}\NormalTok{, }\AttributeTok{lambda =}\NormalTok{ grid) }\CommentTok{\# y is still well{-}being}
\FunctionTok{par}\NormalTok{(}\AttributeTok{mar =} \FunctionTok{c}\NormalTok{(}\DecValTok{5}\NormalTok{, }\DecValTok{4}\NormalTok{, }\DecValTok{4}\NormalTok{, }\DecValTok{8}\NormalTok{), }\AttributeTok{xpd =} \ConstantTok{TRUE}\NormalTok{) }\CommentTok{\# Need space for the legend}
\FunctionTok{plot}\NormalTok{(lasso\_m6s)}
\end{Highlighting}
\end{Shaded}

\begin{verbatim}
Warning in regularize.values(x, y, ties, missing(ties), na.rm = na.rm):
collapsing to unique 'x' values
\end{verbatim}

\begin{Shaded}
\begin{Highlighting}[]
\CommentTok{\# Making the legend}
\NormalTok{lasso\_coefs6 }\OtherTok{\textless{}{-}} \FunctionTok{coef}\NormalTok{(lasso\_m6s)}
\NormalTok{var\_names6 }\OtherTok{\textless{}{-}} \FunctionTok{rownames}\NormalTok{(lasso\_coefs6)[}\SpecialCharTok{{-}}\DecValTok{1}\NormalTok{]}
\NormalTok{colors }\OtherTok{\textless{}{-}} \FunctionTok{seq\_len}\NormalTok{(}\FunctionTok{length}\NormalTok{(var\_names6))}
\FunctionTok{legend}\NormalTok{(}\StringTok{"right"}\NormalTok{, }\AttributeTok{legend =}\NormalTok{ var\_names6, }\AttributeTok{col =}\NormalTok{ colors, }\AttributeTok{lty =} \DecValTok{1}\NormalTok{, }\AttributeTok{cex =} \FloatTok{0.8}\NormalTok{, }\AttributeTok{inset =} \FunctionTok{c}\NormalTok{(}\SpecialCharTok{{-}}\FloatTok{0.25}\NormalTok{, }\DecValTok{0}\NormalTok{))}
\end{Highlighting}
\end{Shaded}

\includegraphics{Episode_2_files/figure-pdf/graph6-1.pdf}

Some of these points make sense to discuss both electoral democracy and
backsliding. Confidence in government quickly increasing in coefficient
once it becomes non-zero reflects overall worries in governments
resulting in increased autocracies. As states' populations trust their
government less, they often turn to the extremes.

Interestingly, the results of Wave 6's correlations are remarkably
gendered. V51, 52, 53, and 208 are all among the top variables
correlated, potentially revealing trends between strict gender norms and
overall level of autocracy. Whether homosexuality is justifiable would
also contribute to this mindset.

The final two variables - importance of leisure time and seeing oneself
as part of the Commonwealth of Independent States (CIS) may be
additional noise, with the latter in particular potentially being due to
the number of autocracies part of the post-Soviet sphere in 2013.

Wave 6 identifies some important correlations in confidence in
government and potential correlations between gendered ideology and
autocracy, in addition to some noise resulting from where autocracies
are. These findings seem to initially confirm an idea that confidence in
governmental institutions is key to democracy, aligning with some
aspects of the literature that find the \emph{lack} of democracy is due
to a lack of trust in institutions. This wave also flags potential
correlations with gender equity as well, as half of the variables it
found as important had to do with if imbalances in gender are justified.

\subsubsection{Wave 7}\label{wave-7}

For the 7th wave, the most important questions are: Q211 \emph{(Openness
to attending a political demonstration)}, Q239 \emph{(Opinion on
religious law governing a country)}, Q38 \emph{(Is it a child's duty to
take care of a sick parent?)}, Q212 \emph{(Openness to joining
unofficial strikes), Q271 (Living with parents)}, Q225 \emph{(Frequency
of opposition candidates prevented from running)}, Q73 \emph{(Confidence
in country's parliament)}, Q3 \emph{(Importance of leisure time)}, Q292I
\emph{(Belief politicians are incompetent or ineffective)}, Q182
\emph{(Is homosexuality justifiable?)}, and Q262 \emph{(Age)}. Many of
these reflect core foundations to democracy, but present novel questions
to ask.

\begin{Shaded}
\begin{Highlighting}[]
\NormalTok{merged7\_select }\OtherTok{\textless{}{-}}\NormalTok{ merged\_lasso }\SpecialCharTok{|\textgreater{}}\NormalTok{ dplyr}\SpecialCharTok{::}\FunctionTok{select}\NormalTok{(}\FunctionTok{names}\NormalTok{(top\_features7))}
\NormalTok{lasso\_m7s }\OtherTok{\textless{}{-}} \FunctionTok{glmnet}\NormalTok{(}\FunctionTok{data.matrix}\NormalTok{(merged7\_select), y7, }\AttributeTok{alpha =} \DecValTok{1}\NormalTok{, }\AttributeTok{lambda =}\NormalTok{ grid) }\CommentTok{\# y is still well{-}being}
\FunctionTok{par}\NormalTok{(}\AttributeTok{mar =} \FunctionTok{c}\NormalTok{(}\DecValTok{5}\NormalTok{, }\DecValTok{4}\NormalTok{, }\DecValTok{4}\NormalTok{, }\DecValTok{8}\NormalTok{), }\AttributeTok{xpd =} \ConstantTok{TRUE}\NormalTok{) }\CommentTok{\# Need space for the legend}
\FunctionTok{plot}\NormalTok{(lasso\_m7s)}
\end{Highlighting}
\end{Shaded}

\begin{verbatim}
Warning in regularize.values(x, y, ties, missing(ties), na.rm = na.rm):
collapsing to unique 'x' values
\end{verbatim}

\begin{Shaded}
\begin{Highlighting}[]
\CommentTok{\# Making the legend}
\NormalTok{lasso\_coefs7 }\OtherTok{\textless{}{-}} \FunctionTok{coef}\NormalTok{(lasso\_m7s)}
\NormalTok{var\_names7 }\OtherTok{\textless{}{-}} \FunctionTok{rownames}\NormalTok{(lasso\_coefs7)[}\SpecialCharTok{{-}}\DecValTok{1}\NormalTok{]}
\NormalTok{colors }\OtherTok{\textless{}{-}} \FunctionTok{seq\_len}\NormalTok{(}\FunctionTok{length}\NormalTok{(var\_names7))}
\FunctionTok{legend}\NormalTok{(}\StringTok{"right"}\NormalTok{, }\AttributeTok{legend =}\NormalTok{ var\_names7, }\AttributeTok{col =}\NormalTok{ colors, }\AttributeTok{lty =} \DecValTok{1}\NormalTok{, }\AttributeTok{cex =} \FloatTok{0.8}\NormalTok{, }\AttributeTok{inset =} \FunctionTok{c}\NormalTok{(}\SpecialCharTok{{-}}\FloatTok{0.25}\NormalTok{, }\DecValTok{0}\NormalTok{))}
\end{Highlighting}
\end{Shaded}

\includegraphics{Episode_2_files/figure-pdf/graph7-1.pdf}

From Wave 7, we can see notable trends in political demonstrations
highly correlating with overall level of electoral democracy; Q211 and
Q212 are among the first variables given coefficients, and Q211 has the
largest overall effect. This makes sense, as generally, an increased
level to take public action for political reasons would increase the
level of accountability leaders have to their people.

Additionally, Wave 7 denotes variables like confidence in parliament,
belief politicians are incompetent or ineffective, and frequency of
opposition candidates being prevented from running. These all directly
tie into whether electoral democracy is healthy, as if opposition
candidates are not allowed to run, overall level of democracy decreases.
Beliefs that the government and its actors are ineffective also tie into
what Wave 6 revealed, as this can decrease trust in democracy being able
to function well.

Interestingly, importance in leisure time and belief that homosexuality
is justifiable were both correlated with electoral democracy level in
Waves 6 and 7, potentially denoting some common social issues
democracies promote. However, they are not as directly tied to the level
of democracy as other variables are.

\subsection{Conclusion}\label{conclusion}

While the global shift to autocracy is bolstered by an overall
frustration in democratic institutions, it is by no means the only
factor. This research confirms the theoretical claim that backsliding is
correlated with an increase in frustrations with democratic
institutions, but also offers other potential variables like democratic
activism and gender equity, which have potential for further theoretical
analysis.

My initial hypothesis of frustration in institutions causing backsliding
seems to hold, as trust in institutions is strongly correlated with
government type. In particular, V115 (that of confidence in government
during Wave 6 of the World Values Survey) was the variable most strongly
correlated with government type for that set of data.

However, Wave 7's results posit an additional variable: that of
democratic activism. Two of the most correlated variables are
specifically about being open to political action. While further
research is required to confirm its causality, this correlation
implicates backsliding not only occurs due to frustration with
governments but also as a result of a low will to protest for political
purposes.

Thus, the level of democracy in a state - at least to some extent - is
affected by the values of its population. If people are dissatisfied
with the government and will rarely take political action, an autocratic
regime is more likely. Using these correlations, we may be able to
predict backsliding for future states, furthering our understanding of
this shift from democracy in an ever-changing world.

\subsection{References}\label{references}

\phantomsection\label{refs}
\begin{CSLReferences}{1}{0}
\bibitem[\citeproctext]{ref-carothersUnderstandingRespondingGlobal2022}
Carothers, Thomas, and Benjamin Press. {``Understanding and {Responding}
to {Global Democratic Backsliding}.''} Carnegie Endowment for
International Peace, October 2022.

\bibitem[\citeproctext]{ref-cooleyExitHegemonyUnraveling2020}
Cooley, Alexander, and Daniel Nexon. \emph{Exit from {Hegemony}: {The
Unraveling} of the {American Global Order}}. Oxford, New York: Oxford
University Press, 2020.

\bibitem[\citeproctext]{ref-haggardBackslidingDemocraticRegress2021}
Haggard, Stephan, and Robert Kaufman. \emph{Backsliding: {Democratic
Regress} in the {Contemporary World}}. Elements in {Political Economy}
9. Cambridge University Press, 2021.

\bibitem[\citeproctext]{ref-haggardAnatomyDemocraticBacksliding2021}
---------. {``The {Anatomy} of {Democratic Backsliding}.''}
\emph{Journal of Democracy} 32, no. 4 (2021): 27--41.

\bibitem[\citeproctext]{ref-levitskyHowDemocracies2018a}
Levitsky, Steven, and Daniel Ziblatt. \emph{How {Democracies Die}}.
First edition. New York: Crown, 2018.

\bibitem[\citeproctext]{ref-meyerroseInternationalSourcesDemocratic2021}
Meyerrose, Anna M. {``International {Sources} of {Democratic
Backsliding}.''} In \emph{Routledge {Handbook} of {Illiberalism}}.
Routledge, 2021.

\bibitem[\citeproctext]{ref-meyerroseUnintendedConsequencesDemocracy2020}
---------. {``The {Unintended Consequences} of {Democracy Promotion}:
{International Organizations} and {Democratic Backsliding}.''}
\emph{Comparative Political Studies} 53, no. 10-11 (September 2020):
1547--81. \url{https://doi.org/10.1177/0010414019897689}.

\bibitem[\citeproctext]{ref-nordDemocracyReport20242024}
Nord, Marina, Martin Lundstedt, David Altman, Fabio Angiolillo, Cecilia
Borella, Tiago Fernandes, Lisa Gastaldi, Ana Good God, Natalia Natsika,
and Staffan Lindberg. {``Democracy {Report} 2024: {Democracy Winning}
and {Losing} at the {Ballot}.''} University of Gothenburg: V-Dem
Institute, March 2024.

\end{CSLReferences}




\end{document}
